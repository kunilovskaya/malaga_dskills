% this is source code for one of the sessions in Digital Skills for Research Workshop (EMTTI, University of Wolverhampton)
% March 2022, Maria Kunilovskaya (mkunilovskaya@gmail.com)

\documentclass[a4paper,11pt]{article}

% custom link command
\usepackage[colorlinks=true, linkcolor=blue, urlcolor=cyan, filecolor=magenta]{hyperref} 

\usepackage{geometry}
\geometry{
	a4paper,
	total={170mm,257mm},
	left=20mm,
	top=15mm,
}
\setlength\parindent{0pt} % set all indents to 0

\usepackage{listings}  % a verbatim environment which can break lines unlike \verb||; load AFTER babel
\usepackage{tcolorbox}
\usepackage{multicol}
\usepackage{todonotes}

\usepackage{graphicx}  % to add graphics
\graphicspath{{images/}{pics/}}  % folders with .png, .jpj, .gif, .eps, .pdf
\usepackage{wrapfig} % put figure inside the text
\usepackage{caption} % automatic names for graphics; default Figure
%\captionsetup{labelsep=period} % add a dot after Figure in captions

%--------------------
% Own commands
% -------------------
\newcommand{\myLaTeX}{\LaTeX~}

\renewcommand*{\figurename}{Fig.}

\newcommand{\boxedfig}[1]{%
	\setlength{\fboxsep}{5pt}%
	\setlength{\fboxrule}{3pt}%
	\fbox{\includegraphics[width=\linewidth]{#1}}%
}

\newenvironment{hello}[1][world]{\noindent Hello #1, }{Bye now!\\} % first [] has number of arguments; second [] has the default value of the first optional argument; second argument is mandatory
\newcommand{\hi}[2][world]{\noindent Hello #1 and #2}

%Numbered environment with double counter-within

\newcounter{example}
\counterwithin*{example}{section} % asterisk/star avoids redefining theexample (second number in 1.2) in each section
\newenvironment{examples}[1][mytitle]{\refstepcounter{example}\par\medskip
	\noindent \textbf{Example~\thesection.\theexample. #1}\par \rmfamily}{\medskip}

%--------------------
% Title
% -------------------

\title{Day 2, part 2. Building from templates}
\author{Digital Skills for Research}
\date{11 May 2022}

\begin{document}
	
	\maketitle
	\tableofcontents


\section{Templates and big projects}
\begin{itemize}
	\item Many publication venues provide their own \LaTeX templates. It makes sense to keep the original source file (.tex) in the project folder for reference. 
	\item Notice lack of extensions in imports.
\end{itemize}

A zipped \LaTeX~template may include the following files (Look into the source code to see another method to produce columns and control their size and positioning):

\medskip

\begin{minipage}[c]{0.65\linewidth}
	\begin{tcolorbox}[title={Computational Linguistics by MIT Press Journals}]
		\begin{description}
			\item[clv3.cls] class file, used in \verb|documentclass{clv3}|
			\item[alocal.sty] style file (a patch for ArabTeX package) to be imported with \verb|\usepackage{alocal}| if necessary
			\item[compling.bst] bibliography style file, used as \verb|\bibliographystyle{compling}|
			\item[compling\_style.bib] example file with bibliography records 
			\item[COLI-manual3.tex] main source code 
		\end{description}
	\end{tcolorbox}
\end{minipage} 
\hfill  % comment out space fill if you would like to put them side by side
\begin{minipage}[c]{0.3\linewidth}
	\begin{tcolorbox}[title={RANLP conference}]
		\begin{itemize}
			\item ranlp2021.sty
			\item acl\_natbib.bst
			\item anthology.bib
			\item ranlp2021.tex
		\end{itemize}
	\end{tcolorbox}
\end{minipage}

\subsection{Principles of working on a multi-part book (like a thesis)}

\begin{itemize}
	\item Adapting an existing template maybe easier than setting up your own from scratch.
	\item Using a template implies familiarity with the packages and commands.
\end{itemize}

\textbf{Specificity of a multi-part book-like project: What is different?}

\begin{itemize}
	\item parts are typeset in separate files which are imported by the main code which defines overall parameters -- e.g. thesis.tex importing 
	
	\begin{lstlisting}
	\include{contents} 
	\include{tables}
	\include{chapters/dedication}
	\include{chapters/1_intro}
	\end{lstlisting}
	
	\item \verb|\documentclass[12pt,a4paper,twoside,openright]{report}|
	\item \textcolor{red}{additional elements in text}:
	
	\begin{itemize}
		\item \verb|\usepackage{epigraph} ... \epigraph{}{}|
		\item headers-footers: 
		\begin{lstlisting}
		\usepackage{fancyhdr}
		\pagestyle{fancy}
		\fancyhead{}
		% RO/LE: position of text on odd/even pages
		\fancyhead[RO,LE]{Thesis Title}	
		\end{lstlisting}
		
		\item \verb|\usepackage{todonotes} ... \todo[inline]{...}|
		
		\item a glossary to collect all abbreviations and acronyms 
		\todo[inline]{This comment is generated by a todo[inline] command: Do I need a glossary?}
		\begin{lstlisting}
		\usepackage[acronym,automake]{glossaries}
		\makeglossaries
		\end{lstlisting}
		
		\item code listings using verbatim \verb|usepackage{listings}|
		
		\begin{lstlisting}
		\lstinputlisting[language=Python]{argpars.py}
		\end{lstlisting}
		
		\lstinputlisting[language=Python, firstline=2, lastline=15, caption=Types of arguments for a Python script from a file]{argpars.py}
		
		\begin{lstlisting}[language=Python, caption=Looping thru all folders under 'root']
		for path, dirs, files in os.walk(args.root):
		last_folder = os.path.abspath(path).split('/')[-1]
		for i, file in enumerate(files):
		filepath = path + os.sep + file
		
		\end{lstlisting}
		
	\end{itemize}
\end{itemize}

\lstlistoflistings


\section*{Task 2-4. Create a document based on a template}
\label{task}
\addcontentsline{toc}{section}{Task 2-4. Create a document based on a template}

\begin{tcolorbox}[width=\textwidth, colback={yellow!40!white}, title={You can complete any of the following tasks. Each involves selecting a template and exploring the commands it uses by modifying it. Acknowledge the original template (add a link to it)in your source code}, colbacktitle=yellow!60!white, coltitle=black]
	\begin{itemize}
		\item Produce a CV using one of the templates \\ \href{here}{https://www.latextemplates.com/cat/curricula-vitae}
		\item Prepare a mock submission to RANLP using their \href{https://www.overleaf.com/latex/templates/instructions-for-ranlp-2021-proceedings/snyphxfdqcpz}{Overleaf template}
		\item Explore and set up a multi-chapter phd/master thesis template based on what is \href{https://github.com/snim2/phdtemplate}{provided by the University of Wolverhampton} and (most recently, in Oct 2021) adapted by \href{https://github.com/TharinduDR/Thesis/}{Tharindu Ranasinghe}
	\end{itemize}
	
\end{tcolorbox}%


\end{document}