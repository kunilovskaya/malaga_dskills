% this is source code for one of the sessions in Digital Skills for Research Workshop (EMTTI, University of Wolverhampton)
% March 2022, Maria Kunilovskaya (mkunilovskaya@gmail.com)

\documentclass[a4paper,11pt]{article}

% custom link command
\usepackage[colorlinks=true, linkcolor=blue, urlcolor=cyan, filecolor=magenta]{hyperref} 

\usepackage{geometry}
\geometry{
	a4paper,
	total={170mm,257mm},
	left=20mm,
	top=15mm,
}
\setlength\parindent{0pt} % set all indents to 0

\usepackage{listings}  % a verbatim environment which can break lines unlike \verb||; load AFTER babel
\usepackage{tcolorbox}
\usepackage{multicol}
\usepackage{todonotes}

\usepackage{graphicx}  % to add graphics
\graphicspath{{images/}{pics/}}  % folders with .png, .jpj, .gif, .eps, .pdf
\usepackage{wrapfig} % put figure inside the text
\usepackage{caption} % automatic names for graphics; default Figure
%\captionsetup{labelsep=period} % add a dot after Figure in captions

%--------------------
% Own commands
% -------------------
\newcommand{\myLaTeX}{\LaTeX~}

\renewcommand*{\figurename}{Fig.}

\newcommand{\boxedfig}[1]{%
	\setlength{\fboxsep}{5pt}%
	\setlength{\fboxrule}{3pt}%
	\fbox{\includegraphics[width=\linewidth]{#1}}%
}

\newenvironment{hello}[1][world]{\noindent Hello #1, }{Bye now!\\} % first [] has number of arguments; second [] has the default value of the first optional argument; second argument is mandatory
\newcommand{\hi}[2][world]{\noindent Hello #1 and #2}

%Numbered environment with double counter-within

\newcounter{example}
\counterwithin*{example}{section} % asterisk/star avoids redefining theexample (second number in 1.2) in each section
\newenvironment{examples}[1][mytitle]{\refstepcounter{example}\par\medskip
	\noindent \textbf{Example~\thesection.\theexample. #1}\par \rmfamily}{\medskip}

%--------------------
% Title
% -------------------

\title{Day 2, part 1. Customisation and Cross-referencing}
\author{Digital Skills for Research}
\date{11 May 2022}

\begin{document}

\maketitle
\tableofcontents

\section{Customisation and own commands}\label{sec:own}


Usually, there are many ways to skin the \hypertarget{wd:random}{cat}.

Levels of customisation: 
class files, style files, packages that provide additional commands and environments, own commands and environments + advanced \texttt{xparse} and \texttt{etoolbox} packages for writing own packages

\begin{itemize}
	\item \verb|\newcommand|: defines a new command; it is a \myLaTeX wrapper on top of \TeX~primitive (\verb|\def|) 
	\begin{itemize}
		\item Adding a space after LaTeX default logo command: \\
		\verb|\newcommand{\myLaTeX}{\LaTeX|$\sim$\}
		\item \verb|\newenvironment{hello}[1][world]{\noindent Hello #1, }{Bye now!\\}|
	\end{itemize}
	\item \verb|\renewcommand|: redefines an existing command
	\begin{itemize}
		\item \verb|\renewcommand{\harvardurl}{URL: \url}|
		\item \verb|\renewcommand{\refname}{Selected Publications 2017-2021}|
		\item \verb|\renewcommand{\figurename}{Fig.}|
		\item (re)new(ed) commands/environments can have [optional] and \{mandatory\} arguments: e.g. \verb|\newcommand{\boxedfig}[1]{...}|
		
	\end{itemize}
	\item modify the default parameters (a) globally for the whole document or (b) locally for parts of it: \\ e.g. 
	\begin{itemize}
		\item put \verb|\setlength\parindent{0pt}| in preamble to cancel all indentation (or \verb|\noindent| for local effect)
		\item \verb|\captionsetup{labelsep=period}| to use a dot (not colon) after Fig(ure) 1 in captions
		\item \verb|\thispagestyle{empty}| on any page to lose the page number (see page \pageref{pg:empty} in this document)
		\item adding space, changing fonts and text alignment locally with existing commands: \\ 
		\verb|One {\Large{word}} appears large|
	\end{itemize}
	
\end{itemize}

\clearpage

{\centering 
	
	\textbf{Here are a few simple examples \\ Notice and explore the numbering of the examples linked to Sections}
	
}
\begin{examples}
	This is some text with the default \LaTeX command. \par
	And this sentence calls the modi	content...
\end{document}fied \myLaTeX command. \par
	(Notice the added space after the logo.)
\end{examples}

\begin{wrapfigure}{r}{0.3333\linewidth}
	\boxedfig{lines.eps}
	\caption{Two lines plot in a box}
	\label{fig:logo}
\end{wrapfigure}


\begin{examples} 
	\par
	Two calls of the hello environment:
	\begin{lstlisting}
	\begin{hello}
	nice to meet you.
	\end{hello}
	
	\begin{hello}[Bob]
	glad you could make it.
	\end{hello}
	\end{lstlisting}
	
	Output:\\
	\begin{hello}
		nice to meet you.
	\end{hello}
	\begin{hello}[Bob]
		glad you could make it.
	\end{hello}
\end{examples}



\begin{examples}
	\begin{lstlisting}
		\renewcommand*{\figurename}{Fig.}
		
		\newcommand{\boxedfig}[1]{%
		\setlength{\fboxsep}{5pt}%
		\setlength{\fboxrule}{3pt}%
		\fbox{\includegraphics[width=\linewidth]{#1}}%
		}
	\end{lstlisting}
	
	Called as:
	\begin{lstlisting}
		\begin{wrapfigure}{r}{0.3333\linewidth}
			\boxedfig{lines.eps}
			\caption{Two lines plot in a box}
			\label{fig:logo}
		\end{wrapfigure}
	\end{lstlisting}
\end{examples}

\begin{examples}
	\verb|\newcommand{\hi}[2][world]{\noindent Hello #1 and #2}| \\
	called as 	\verb|\hi[Marie][Stephen]| and as \verb|\hi{Stephen}|\\
	
	\hi[Marie]{Stephen}
	
	\hi{Stephen}
\end{examples}

\textcolor{red}{NB!} Asterisks in commands definitions and per cent signs at the end of lines are safety checks to prevent arguments accedentally containing blank lines or \verb|\par|.

\bigskip
\textcolor{red}{NB!} Renewing commands that have \verb|@| in their name requires:
\begin{lstlisting}
\makeatletter
\renewcommand*{\verbatim@font}{March 11, \ttfamily\footnotesize}
\makeatother
\end{lstlisting}
This sort of redefinition cannot be used in .sty files.

\clearpage

\section{Internal and external links}\label{sec:links}

The main package to allow cross-referencing is \verb|\usepackage{hyperref}|.

Types of links:

\begin{itemize}
	\item Internal links (inc. to individual words): \verb|\label{sec:links} ... \ref{sec:links}|
	
	In Section~\ref{sec:links} we used \ldots
	
	\item Links to local files: \\ \verb|\href{run:./pics/Pym_2020_translation_solutions_ES>EN.pdf}{Pym's paper (2020)}| 
	
	See \href{run:./pics/Pym_2020_translation_solutions_ES>EN.pdf}{Pym's paper (2020)}
	
	
	\item Web addresses: \verb|\href{https://en.wikipedia.org/wiki/LaTeX}{Wiki on Latex}| \\
and \verb|\url{https://en.wikipedia.org/wiki/LaTeX}| 

	This is what Wikipedia says about Latex: \href{https://en.wikipedia.org/wiki/LaTeX}{Wiki on Latex} or with visible address \url{https://en.wikipedia.org/wiki/LaTeX}

\end{itemize}

\textbf{Custom colours for each type of links (seems to be a paper-friendly solution)}

\begin{lstlisting}[breaklines]
	\hypersetup{
		colorlinks=true,
		linkcolor=blue,
		filecolor=magenta,      
		urlcolor=cyan,
	}
\end{lstlisting}

% ; load early in the preamble as some packages complain 
To refer back to a particular word/phrase in the document, use: 
\begin{lstlisting}[breaklines]
\hypertarget{wd:cats}{where_to_return} ...
 \hyperlink{wd:cats}{word_to_make_clickable}
\end{lstlisting}

\thispagestyle{empty}\label{pg:empty}

In Section~\ref{sec:own}, we talked about some \hyperlink{wd:random}{cats}.

\bigskip

Hide all the clickables (good for printing on paper, but not for an e-document): \\
\verb|\usepackage[hidelinks]{hyperref}|



\end{document}