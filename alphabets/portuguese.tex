\documentclass{article}

%encoding
%--------------------------------------
\usepackage[T1]{fontenc}
\usepackage[utf8]{inputenc}
%--------------------------------------

%Portuguese-specific commands
%--------------------------------------
\usepackage[portuguese]{babel}
%--------------------------------------

%Hyphenation rules
%--------------------------------------
\usepackage{hyphenat}
\hyphenation{mate-mática recu-perar}
%--------------------------------------

\begin{document}

\tableofcontents

\vspace{2cm} %Add a 2cm space

\begin{abstract}
Este é um breve resumo do conteúdo do documento escrito em Português.
\end{abstract}

\section{Seção introdutória}
Esta é a primeira seção, podemos acrescentar alguns elementos adicionais 
e tudo será escrito corretamente. Além disso, se uma palavra é um caminho 
muito longo e tem de ser truncado, babel irá tentar truncar corretamente, 
dependendo do idioma.

\section{Segunda seção}
Esta seção é para ver o que acontece com comandos de texto que definem

\[ \lim x =  \theta + 152383.52 \]

\end{document}