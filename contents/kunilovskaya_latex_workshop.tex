% this is source code for one of the sessions in Digital Skills for Research Workshop (EMTTI, University of Wolverhampton)
% see an updated and restructured version of the workshop as run in Malaga, 10-12 May 2022 saved as a malaga_dskills COPY of the repository
% March 2022, Maria Kunilovskaya (mkunilovskaya@gmail.com)

%--------------------
% Preamble

% Declare the type of document
% -------------------

\documentclass[a4paper,12pt]{article} % other options: [,twocolumn,leqno]{report, book, beamer} 

%--------------------
% Import packages
% -------------------

\usepackage[utf8]{inputenc}  % specify the encoding
\usepackage[T1]{fontenc}
\usepackage{hyperref} % allow cross-referencing

\usepackage{geometry} % set the layout
\geometry{
	a4paper,
	total={170mm,257mm},
	left=20mm,
	top=20mm,
}
\usepackage{xcolor} % alow color for text
\usepackage{tcolorbox} % make boxes
\usepackage{multicol} % columns

\usepackage{tipa} % allow use of phonetic notation
\usepackage{newunicodechar}
%\newunicodechar{a}{ɑ}
%\newunicodechar{ː}{:}
\usepackage{graphicx} % insert pictures
%\graphicspath{{images/}{pics/}}  % from folders with .png, .jpj, .gif, .eps

%--------------------
% Define new commands and settings
% -------------------

% TeX logo as defined by Donald Knuth in the TeXbook (1984)
\def\TeX{{\rm T\kern-.1667em\lower.5ex\hbox{E}\kern-.125emX }}
\newcommand{\llogo}{\LaTeX}

\setlength\parindent{0pt} % don't indent new paragraphs

%--------------------
% Setup the title of the document
% -------------------

\title{\vspace{-4em} Digital Skills for Research \\ \LaTeX, Mendeley/Zotero and Github}
\author{Maria Kunilovskaya}
\date{%
	University of Malaga, EM TTI Cohorts 2 and 3\\%
	---\\%
	10-12 May, 2022}

\begin{document}
	
%remove numbering from the first page
\clearpage\maketitle
\thispagestyle{empty}	
\maketitle

\vspace{-2em}

\section*{Day 1. Setup, basic commands and environments}
\begin{enumerate}
	\item Why \TeX and first doc 
		\begin{itemize}
			\item What's \TeX (distributions, editors, engines, formats, templates)
			\item Overleaf account and project
			\item First document (class, packages, layout, title, sections, margins, columns)
	\end{itemize} 
	\item Text and math
		\begin{itemize}
			\item Text formatting
			\item Special characters
			\item Math
		\end{itemize}
	\item Lists, tables and figures
		\begin{itemize}
			\item Environments
			\item Tables
			\item Graphics and drawing
		\end{itemize}
\end{enumerate}
\section*{Day 2. Cross-refs, templates and beamer}
\begin{enumerate}
	\item Cross-referencing and customisation 
		\begin{itemize}
			\item Internal and external links
			\item Own commands
		\end{itemize}
	\item Building from templates
		\begin{itemize}
			\item Templates and big projects
			\item Preliminaries on bibliographic references
		\end{itemize}
	\item Presentations and posters 
		\begin{itemize}
			\item Beamer themes: layout and colour
			\item Slides-specific commands
			\item Producing posters and own .sty 
		\end{itemize}
\end{enumerate}
	
\section*{Day 3. Bibliography and Github}
\begin{enumerate}
	\item Reference (citation) styles
		\begin{itemize}
			\item Morphology of a bibliographic description (APA, Chicago and Harvard)
			\item .bib files and bibliography engines
			\item In-text and end-of-text referencing commands
		\end{itemize}%
	\item Reference management: Mendeley or Zotero
		\begin{itemize}
			\item Basic uses and setting up
			\item Integration (word processor, browser, Latex project)
			\item Library use and maintenance 
		\end{itemize}%
	\item Version control and collaboration: Git and GitHub
		\begin{itemize}
			\item Keeping track of changes
			\item Local and remote, push and pull, auth
			\item Markdown and arranging repos
		\end{itemize}
\end{enumerate}
\begin{center}

\begin{tcolorbox}[width=\textwidth, title={\textbf{Recommended resources}}, colbacktitle=white, coltitle=black, colback={white}]
	\begin{itemize}
	\item \textbf{LaTeX}: \href{https://tug.org/begin.html}{\TeX UsersGroup} and \href{https://www.overleaf.com/learn/latex/Learn_LaTeX_in_30_minutes}{Overleaf} tutorials
	\item \textbf{GitHub}: \href{https://git-scm.com/book/en/v2}{Pro Git book by Scott Chacon and Ben Straub} and \href{https://docs.github.com/en/get-started}{Get started from github.com}
	\item \textbf{Mendeley}: \href{https://www.mendeley.com/guides}{Official Guides}
	
\end{itemize}
	
\end{tcolorbox}%

\bigskip

\begin{tcolorbox}[width=\textwidth, title={\textbf{Historical Notes and Trivia}}, colbacktitle=yellow!20, coltitle=black, colback={yellow!20}]
	\begin{itemize}
		\item \TeX is a typesetting system which was designed by Donald Knuth and first released in 1978.
		\item The letters of the name are meant to represent the capital Greek letters tau, epsilon, and chi ($\tau\epsilon\chi$), Greek for both ``art'' and ``craft'' (\textit{tech}nology).
		\item The word \LaTeX is an abbreviation of ``Lamport's TeX'', named after Leslie Lamport. He added and described \textbf{a collection of macros} (1986) to the original \TeX.
		\item  pronounced as \textipa{/"la:tEk/} or \textipa{/"leItEk/} 
		\item It is also a pun with the English word \textit{latex} (try it in image google search)
	\end{itemize}
	
\end{tcolorbox}%

\end{center}

\end{document}